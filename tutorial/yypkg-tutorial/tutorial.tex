\documentclass[a4paper]{article}

\usepackage[utf8]{inputenc}
\usepackage{graphicx}

% \pdfimageresolution=108

\author{Adrien Nader}
\title{Introduction tutorial to Yellow Yeti Pkg, a fast, portable and inobtrusive package manager}

\begin{document}

\maketitle

\tableofcontents
\section{Installation}
\subsection{Dependencies}
Currently, yypkg depends on:
\begin{itemize}
  \setlength{\parskip}{0em}
  \item{ocaml-fileutils}: a library that provides the functionality of the coreutils package under unices.
  \item{sexplib}: automated creation of type-safe serialization/deserialization functions from type definitions. All serialization functions in yypkg rely on it.
  \item{lablgtk2}: used for the gui, not a strong dependency.
\end{itemize}

\subsection{Windows}
\subparagraph{} It is advised to build under windows with the 'preprocessed\_src' folder and the 'BUILDING.windows' script (chmod +x it).

\subparagraph{} Support for windows is a bit "dirty". That doesn't mean it's bad, it only means that fileutils and sexplib have to be put in \$(ocamlc -where) and lablgtk2 in \$(ocamlc -where)/lablgtk.\\
Also, findlib is not supported (the usual 'make' should work if you have everything setup, including camlp4, of course).

\section{YYPREFIX and -prefix}
Since yypkg works in its own prefix, you need to tell it its value. This can be done in two ways: the YYPREFIX environment variable, or the -prefix switch to command-line arguments. It is advised to use YYPREFIX.

With most shells, run: export YYPREFIX=/path/to/your/prefix. The path can be absolute or relative.

On windows, in cmd.exe, run: set YYPREFIX=C:/path/to/your/prefix. Both forward and backward slashes work. You can also set it globally in System Properties.

From now on, we assume that you've set YYPREFIX.

\section{Initialisation}
\subsection{-init}
First, run yypkg -init. This will install some files, mostly create default settings and an empty package database.

On windows, it will also copy binaries to sbin/ so these can always be found (wget, bsdtar, liblzma...).

\begin{verbatim}
~/yypkg % export YYPREFIX=$(pwd)/prefix
~/yypkg % ./src/yypkg.native -init
~/yypkg % find prefix \! -type d | xargs ls -lh         
-rw-r--r-- 1 tux users 134 Dec 26 22:17 prefix/etc/yypkg.d/sherpa.conf
-rw-r--r-- 1 tux users  12 Dec 26 22:17 prefix/etc/yypkg.d/yypkg.conf
-rw-r--r-- 1 tux users   2 Dec 26 22:17 prefix/var/log/packages/yypkg_db
\end{verbatim}

% \subsection{-config -setpreds}
% Now, you have to tell yypkg which architectures it should accept for the packages. For instance, for win32 packages, you'll use 'i686-w64-mingw32'.
% \begin{verbatim}
% ~/yypkg % ./src/yypkg.native -config -setpreds arch=noarch,i686-w64-mingw32
% \end{verbatim}
% 
% And for win64 packages (there are no win64 packages currently so don't use yet):
% 
% \begin{verbatim}
% ~/yypkg % ./src/yypkg.native -config -setpreds arch=noarch,x86_64-w64-mingw32
% \end{verbatim}
% 
% \section{sherpa-gui}
% All the functions in yypkg are accessible from command-line and a graphical interface named sherpa (the sherpa carries packages to the yeti...).
% 
% \begin{center}
%   \includegraphics{../screenshots/sherpa_gui_1.png}
% \end{center}
% 
% \subsection{Starting}
% Currently, you should always start sherpa\_gui from a terminal since it'll write a lot to stdout/stderr. Also, you have to start it with the -sherpa parameter:
% \begin{verbatim}
% sherpa_gui -sherpa
% \end{verbatim}
% 
% \subsection{Getting the package list}
% Once it is started, go to Package List -> Force Update. You will be presented with a list of packages in a column layout. Most things are obvious so we'll only explain what goes with the 'Selected' and 'Deps included' columns.
% 
% \subsection{One word about package selection and dependencies}
% YYPkg uses a very simple algorithm to manage dependencies.  Here, 'Selected' means that {\bf you} selected a package. When you do, the 'Deps included' column is update to reflect which packages the items that you selected depend on.\\
% If you do not want a package that is ticked in the 'Deps included' column, simply untick it.\\
% 
% The dependencies are merely suggested, there is nothing to guarantee that all dependencies are met and nothing to 'force' one of them.
% 
% During uninstallation, the dependencies are simply not involved at all.
% 
% \subsection{Processing}
% Once you are done, with the selection, go to File -> Process. Packages will be downloaded, installed for the ones that are selected and uninstalled for those that have to be.
% 
% There is currently no support for updating a package: uninstall the old one and install the new one. This will be fixed shortly.

\section{Command-line tools: yypkg and sherpa}
These tools are not described in this document: see the 'USAGE' file:
<a href="http://git.ocamlcore.org/cgi-bin/gitweb.cgi?p=yypkg/yypkg.git;a=blob;f=USAGE">pouet</a>.
\end{document}

